In Exercises \ref{usefuncgraphfirst} - \ref{usefuncgraphlast}, use the graph of $y = f(x)$ given below to answer the  question.


\begin{center}

\begin{mfpic}[15]{-6}{6}{-6}{6}
\axes
\tlabel[cc](6,-0.5){\scriptsize $x$}
\tlabel[cc](0.5,6){\scriptsize $y$}
\tlabel[cc](-5,-5.5){\scriptsize $(-5,-5)$}
\tlabel[cc](-5,0.5){\scriptsize $(-4,0)$}
\tlabel[cc](-3,4.5){\scriptsize $(-3,4)$}
\tlabel[cc](-1,2){\scriptsize $(-2,2)$}
\tlabel[cc](-2,-0.5){\scriptsize $(-1,0)$}
\tlabel[cc](1,-1.5){\scriptsize $(0,-1)$}
\tlabel[cc](1.75,-0.5){\scriptsize $(1,0)$}
\tlabel[cc](2,3.5){\scriptsize $(2,3)$}
\tlabel[cc](4,1){\scriptsize $(3,1)$}
\xmarks{-5,-4,-3,-2,-1,1,2,3,4,5}
\ymarks{-5,-4,-3,-2,-1,1,2,3,4,5}
\tlpointsep{5pt}
\scriptsize
\axislabels {x}{{$-5 \hspace{7pt}$} -5,  {$3$} 3, {$4$} 4, {$5$} 5}
\axislabels {y}{{$-5$} -5,{$-4$} -4,{$-3$} -3,{$-2$} -2, {$1$} 1, {$3$} 3, {$4$} 4, {$5$} 5}
\normalsize
\point[4pt]{(-5, -5), (-4, 0), (-3, 4), (-2,2), (-1,0), (0,-1), (1,0), (2,3), (3,1)}
\penwd{1.25pt}
\polyline{(-5,-5), (-4,0), (-3,4), (-2,2), (-1,0)}
\function{-1, 1, 0.1}{(x**2)-1}
\polyline{(1,0), (2,3), (3,1)}
\tcaption{$y=f(x)$}
\end{mfpic}

\end{center}


\begin{multicols}{2}
\begin{enumerate}


\item  Find the domain of $f$. \label{usefuncgraphfirst}
\item  Find the range of $f$.

\setcounter{HW}{\value{enumi}}
\end{enumerate}
\end{multicols}

\begin{multicols}{2}
\begin{enumerate}
\setcounter{enumi}{\value{HW}}

\item  Find the maximum, if it exists.
\item  Find the minimum, if it exists. \label{usefuncgraphlast}

\setcounter{HW}{\value{enumi}}
\end{enumerate}
\end{multicols}

\begin{multicols}{2}
\begin{enumerate}
\setcounter{enumi}{\value{HW}}

\item  List the local maximums, if any exist.
\item  List the local minimums, if any exist.

\setcounter{HW}{\value{enumi}}
\end{enumerate}
\end{multicols}

\begin{multicols}{2}
\begin{enumerate}
\setcounter{enumi}{\value{HW}}

\item  List the intervals where $f$ is increasing.
\item  List the intervals where $f$ is decreasing.

\setcounter{HW}{\value{enumi}}
\end{enumerate}
\end{multicols}


\begin{multicols}{2}
\begin{enumerate}
\setcounter{enumi}{\value{HW}}

\item  Determine $f(-2)$.
\item  Solve $f(x) = 4$.

\setcounter{HW}{\value{enumi}}
\end{enumerate}
\end{multicols}

\begin{multicols}{2}
\begin{enumerate}
\setcounter{enumi}{\value{HW}}

\item  List the $x$-intercepts, if any exist.
\item  List the $y$-intercepts, if any exist.

\setcounter{HW}{\value{enumi}}
\end{enumerate}
\end{multicols}

\begin{multicols}{2}
\begin{enumerate}
\setcounter{enumi}{\value{HW}}

\item  Find the zeros of $f$.
\item  Solve $f(x) \geq 0$.

\setcounter{HW}{\value{enumi}}
\end{enumerate}
\end{multicols}

\begin{multicols}{2}
\begin{enumerate}
\setcounter{enumi}{\value{HW}}

\item  Find the number of solutions to $f(x) = 1$.
\item   Find the number of solutions to $|f(x)| = 1$.

\setcounter{HW}{\value{enumi}}
\end{enumerate}
\end{multicols}




\begin{multicols}{2}
\begin{enumerate}
\setcounter{enumi}{\value{HW}}

\item   Solve $(x^2-x-2)f(x) = 0$
\item   Solve  $(x^2-x-2)f(x) > 0$

\setcounter{HW}{\value{enumi}}
\end{enumerate}
\end{multicols}

With help from your classmates:

\begin{multicols}{2}
\begin{enumerate}
\setcounter{enumi}{\value{HW}}

\item   Find the domain of $R(x) = \dfrac{1}{f(x)}$
\item   Find the range of $R(x) = \dfrac{1}{f(x)}$

\setcounter{HW}{\value{enumi}}
\end{enumerate}
\end{multicols}


\newpage

In Exercises \ref{usesecondfuncgraphfirst} - \ref{usesecondfuncgraphlast}, use the graph of $y =g(t)$ given below to answer the  question.


\begin{center}

\begin{mfpic}[15]{-5}{5}{-6}{6}
\axes
\tlabel[cc](5,-0.5){\scriptsize $t$}
\tlabel[cc](0.5,6){\scriptsize $y$}
\tlabel[cc](-4,0.5){\scriptsize $(-4,0) \hspace{6pt}$}
\tlabel[cc](-2,-5.5){\scriptsize $(-2,-5) \hspace{6pt}$}
\tlabel[cc](1,-0.5){\scriptsize $(0,0)$}
\tlabel[cc](2,2.5){\scriptsize $(2,3)$}
\tlabel[cc](2,5.5){\scriptsize $(2,5)$}
\tlabel[cc](4,-0.5){\scriptsize $(4,0)$}
\xmarks{-3,-2,-1,1,2,3,4}
\ymarks{-5,-4,-3,-2,-1,1,2,3,4,5}
\tlpointsep{5pt}
\scriptsize
\axislabels {x}{{$-3 \hspace{7pt}$} -3,{$-2 \hspace{7pt}$} -2, {$-1 \hspace{7pt}$} -1}
\axislabels {y}{{$-5$} -5,{$-4$} -4,{$-3$} -3,{$-1$} -1, {$1$} 1, {$2$} 2, {$3$} 3, {$4$} 4, {$5$} 5}
\normalsize
\penwd{1.25pt}
\function{-4, 4, 0.1}{5*sin(x*3.14159/4)}
\point[4pt]{ (-2, -5), (0, 0), (4,0), (2,3), (-4,0)}
\pointfillfalse
\point[4pt]{(2,5)}
\tcaption{$y=g(t)$}
\end{mfpic}

\end{center}

\begin{multicols}{2}
\begin{enumerate}
\setcounter{enumi}{\value{HW}}

\item  Find the domain of $g$. \label{usesecondfuncgraphfirst}
\item  Find the range of $g$.

\setcounter{HW}{\value{enumi}}
\end{enumerate}
\end{multicols}

\begin{multicols}{2}
\begin{enumerate}
\setcounter{enumi}{\value{HW}}

\item  Find the maximum, if it exists.
\item  Find the minimum, if it exists. \label{usesecondfuncgraphlast}

\setcounter{HW}{\value{enumi}}
\end{enumerate}
\end{multicols}

\begin{multicols}{2}
\begin{enumerate}
\setcounter{enumi}{\value{HW}}

\item  List the local maximums, if any exist.
\item  List the local minimums, if any exist.

\setcounter{HW}{\value{enumi}}
\end{enumerate}
\end{multicols}

\begin{multicols}{2}
\begin{enumerate}
\setcounter{enumi}{\value{HW}}

\item  List the intervals where $g$ is increasing.
\item  List the intervals where $g$ is decreasing.

\setcounter{HW}{\value{enumi}}
\end{enumerate}
\end{multicols}


\begin{multicols}{2}
\begin{enumerate}
\setcounter{enumi}{\value{HW}}

\item  Determine $g(2)$.
\item  Solve $g(t) = -5$.

\setcounter{HW}{\value{enumi}}
\end{enumerate}
\end{multicols}

\begin{multicols}{2}
\begin{enumerate}
\setcounter{enumi}{\value{HW}}

\item  List the $t$-intercepts, if any exist.
\item  List the $y$-intercepts, if any exist.

\setcounter{HW}{\value{enumi}}
\end{enumerate}
\end{multicols}

\begin{multicols}{2}
\begin{enumerate}
\setcounter{enumi}{\value{HW}}

\item  Find the zeros of $g$.
\item  Solve $g(t) \leq 0$.

\setcounter{HW}{\value{enumi}}
\end{enumerate}
\end{multicols}

\begin{multicols}{2}
\begin{enumerate}
\setcounter{enumi}{\value{HW}}

\item  Find the domain of $G(t) = \dfrac{g(t)}{t+2}$.
\item  Solve $\dfrac{g(t)}{t+2} \leq 0$.

\setcounter{HW}{\value{enumi}}
\end{enumerate}
\end{multicols}

\begin{multicols}{2}
\begin{enumerate}
\setcounter{enumi}{\value{HW}}

\item  How many solutions are there to $[g(t)]^2 = 9$?
\item  Does $g$ appear to be even, odd, or neither?

\setcounter{HW}{\value{enumi}}
\end{enumerate}
\end{multicols}

\begin{enumerate}
\setcounter{enumi}{\value{HW}}

\item  Prove that if $f$ is an odd function and $0$ is in the domain of $f$, then $f(0) = 0$.

\item  Let $R(x)$ be the function defined as:  $R(x) = 1$ if $x$ is a rational number, $R(x) = 0$ if $x$ is an irrational number. With help from your classmates, try to graph $R$.  What difficulties do you encounter?

NOTE:  Between every pair of real numbers, there is both a rational and an irrational number \ldots


\setcounter{HW}{\value{enumi}}
\end{enumerate}

\newpage

\begin{enumerate}
\setcounter{enumi}{\value{HW}}

\item Consider the graph of the function $f$ given below.  

\begin{center}

\begin{mfpic}[15]{-3}{3}{-3.5}{4}
\axes
\tlabel[cc](3,-0.5){\scriptsize $x$}
\tlabel[cc](0.5,3.75){\scriptsize $y$}
\tlabel[cc](2,1){\scriptsize $(1,1)$}
\tlabel[cc](-2.25,1){\scriptsize $(-1,1)$}
\xmarks{-2,-1,1,2}
\ymarks{-3,-2,-1,1,2,3}
\tlpointsep{5pt}
\scriptsize
\axislabels {x}{{$-2 \hspace{7pt}$} -2, {$-1 \hspace{7pt}$} -1, {$1$} 1, {$2$} 2}
\axislabels {y}{{$-3$} -3,{$-2$} -2,{$-1$} -1,  {$2$} 2, {$3$} 3}
\normalsize
\point[4pt]{(-1, 1), (0, 1), (1, 1)}
\penwd{1.25pt}
\arrow \reverse \function{-3,-1,0.1}{2*x + 3}
\polyline{(-1, 1), (1,1)}
\arrow \function{1,3,0.1}{x}
\end{mfpic}

\end{center}

\begin{enumerate}

\item Explain why $f$ has a local maximum but not a local minimum at the point $(-1, 1)$.

\item Explain why  $f$ has a local minimum but not a local maximum at the point $(1, 1)$.

\item Explain why $f$ has a local maximum AND a local minimum at the point $(0, 1)$.

\item Explain why $f$ is constant on the interval $[-1, 1]$ and thus has both a local maximum AND a local minimum at every point $(x, f(x))$ where $-1 < x < 1$.

\end{enumerate}

\item Explain why  the function $g$ whose graph is given below does \underline{not} have a local maximum at $(-3, 5)$ nor does it have a local minimum at $(3, -3)$.  Find its extrema, both local and absolute and find the intervals on which $g$ is increasing and those on which $g$ is decreasing.

\begin{center}

\begin{mfpic}[15]{-4}{4}{-4.5}{6}
\axes
\tlabel[cc](4,-0.5){\scriptsize $x$}
\tlabel[cc](0.5,5.75){\scriptsize $y$}
\tlabel[cc](-3,5.5){\scriptsize $(-3,5) \hspace{6pt}$}
\tlabel[cc](2,0.5){\scriptsize $(2,0)$}
\tlabel[cc](-1.5,-4){\scriptsize $(0,-4)$}
\tlabel[cc](3,-3.5){\scriptsize $(3,-3)$}
\xmarks{-3,-2,-1,1,2,3}
\ymarks{-4,-3,-2,-1,1,2,3,4,5}
\tlpointsep{5pt}
\scriptsize
\axislabels {x}{{$-3 \hspace{7pt}$} -3,  {$-1 \hspace{7pt}$} -1, {$1$} 1, {$3$} 3}
\axislabels {y}{{$-3$} -3,{$-2$} -2,{$-1$} -1, {$1$} 1, {$2$} 2, {$3$} 3, {$4$} 4, {$5$} 5}
\normalsize
\point[4pt]{(-3,5), (0,-4), (2,0), (3,-3)}
\penwd{1.25pt}
\function{-3,2,0.1}{x**2 - 4}
\polyline{(2,0), (3, -3)}
\end{mfpic}

\end{center}

\setcounter{HW}{\value{enumi}}
\end{enumerate}


\newpage

\begin{enumerate}
\setcounter{enumi}{\value{HW}}



\item  For each function below, find the local maximum or local minimum and list the interval over which the function is increasing and the interval over which the function is decreasing. 

\begin{multicols}{2}
\begin{enumerate}

\item \begin{mfpic}[15]{-3}{3}{-2}{5}
\axes
\tlabel[cc](3,-0.5){\scriptsize $x$}
\tlabel[cc](0.5,5){\scriptsize $y$}
\xmarks{-2,-1,1,2}
\ymarks{-1,1,2,3,4}
\tcaption{Function I}
\tlpointsep{5pt}
\scriptsize
\axislabels {x}{{$-2 \hspace{7pt}$} -2, {$-1 \hspace{7pt}$} -1, {$1$} 1, {$2$} 2}
\axislabels {y}{{$-1$} -1, {$1$} 1, {$2$} 2, {$3$} 3, {$4$} 4}
\normalsize
\point[4pt]{(-2,4), (0, 1), (2,4)}
\penwd{1.25pt}
\function{-2,2,0.1}{x**2}
\pointfillfalse
\point[4pt]{(0,0)}
\end{mfpic}

\item \begin{mfpic}[15]{-3}{3}{-2}{5}
\axes
\tlabel[cc](3,-0.5){\scriptsize $x$}
\tlabel[cc](0.5,5){\scriptsize $y$}
\xmarks{-2,-1,1,2}
\ymarks{-1,1,2,3,4}
\tcaption{Function II}
\tlpointsep{5pt}
\scriptsize
\axislabels {x}{{$-2 \hspace{7pt}$} -2, {$-1 \hspace{7pt}$} -1, {$1$} 1, {$2$} 2}
\axislabels {y}{{$-1$} -1, {$1$} 1, {$2$} 2, {$3$} 3, {$4$} 4}
\normalsize
\penwd{1.25pt}
\point[4pt]{(-2,0), (0, 1), (2,0)}
\function{-2,2,0.1}{4 - x**2}
\pointfillfalse
\point[4pt]{(0,4)}
\end{mfpic}

\setcounter{HWindent}{\value{enumii}}
\end{enumerate}
\end{multicols}

\begin{multicols}{2}
\begin{enumerate}
\setcounter{enumii}{\value{HWindent}}

\item \begin{mfpic}[15]{-3}{3}{-2}{5}
\axes
\tlabel[cc](3,-0.5){\scriptsize $x$}
\tlabel[cc](0.5,5){\scriptsize $y$}
\xmarks{-2,-1,1,2}
\ymarks{-1,1,2,3,4}
\tcaption{Function III}
\tlpointsep{5pt}
\scriptsize
\axislabels {x}{{$-2 \hspace{7pt}$} -2, {$-1 \hspace{7pt}$} -1, {$1$} 1, {$2$} 2}
\axislabels {y}{{$-1$} -1, {$1$} 1, {$2$} 2, {$3$} 3, {$4$} 4}
\normalsize
\penwd{1.25pt}
\point[4pt]{(-2,4), (0, -1), (2,4)}
\function{-2,2,0.1}{x**2}
\pointfillfalse
\point[4pt]{(0,0)}
\end{mfpic}

\item \begin{mfpic}[15]{-3}{3}{-1}{6}
\axes
\tlabel[cc](3,-0.5){\scriptsize $x$}
\tlabel[cc](0.5,6){\scriptsize $y$}
\xmarks{-2,-1,1,2}
\ymarks{1,2,3,4,5}
\tcaption{Function IV}
\tlpointsep{5pt}
\scriptsize
\axislabels {x}{{$-2 \hspace{7pt}$} -2, {$-1 \hspace{7pt}$} -1, {$1$} 1, {$2$} 2}
\axislabels {y}{{$1$} 1, {$2$} 2, {$3$} 3, {$4$} 4, {$5$} 5}
\normalsize
\penwd{1.25pt}
\point[4pt]{(-2,0), (0, 5), (2,0)}
\function{-2,2,0.1}{4 - x**2}
\pointfillfalse
\point[4pt]{(0,4)}
\end{mfpic}

\end{enumerate}

\end{multicols}

\end{enumerate}



\newpage

\subsection{Answers}

\begin{multicols}{3}
\begin{enumerate}

\item  $[-5,3]$
\item  $[-5,4]$
\item  $f(-3) = 4$


\setcounter{HW}{\value{enumi}}
\end{enumerate}
\end{multicols}

\begin{multicols}{3}
\begin{enumerate}
\setcounter{enumi}{\value{HW}}


\item  $f(-5) = -5$
\item  $(-3,4)$,  $(2,3)$
\item  $(0,-1)$

\setcounter{HW}{\value{enumi}}
\end{enumerate}
\end{multicols}



\begin{multicols}{3}
\begin{enumerate}
\setcounter{enumi}{\value{HW}}


\item  $[-5,-3]$, $[0,2]$
\item  $[-3,0]$, $[2,3]$
\item  $f(-2) = 2$


\setcounter{HW}{\value{enumi}}
\end{enumerate}
\end{multicols}


\begin{multicols}{3}
\begin{enumerate}
\setcounter{enumi}{\value{HW}}

\item  $x=-3$
\item $(-4,0)$, $(-1,0)$, $(1,0)$
\item  $(0,-1)$

\setcounter{HW}{\value{enumi}}
\end{enumerate}
\end{multicols}

\begin{multicols}{3}
\begin{enumerate}
\setcounter{enumi}{\value{HW}}

\item  $-4$, $-1$, $1$
\item  $[-4,-1]$, $[1,3]$
\item  $4$

\setcounter{HW}{\value{enumi}}
\end{enumerate}
\end{multicols}


\begin{multicols}{3}
\begin{enumerate}
\setcounter{enumi}{\value{HW}}

\item  $6$
\item  $x=-4, -1,1,2$
\item  $(-4,-1) \cup (-1,1) \cup (2,3)$ 

\setcounter{HW}{\value{enumi}}
\end{enumerate}
\end{multicols}

\begin{enumerate}
\setcounter{enumi}{\value{HW}}

\item To find the domain of $R(x) = \frac{1}{f(x)}$, we start with the domain of $f$ and exclude values where $f(x) = 0$.  Hence, the domain of $R$ is $[-5,-4) \cup (-4,-1) \cup (-1,1) \cup (1,3]$.

\item  To find the range of $R(x) = \frac{1}{f(x)}$, we start with the range of $f$ (excluding $0$)  and take reciprocals.  If $-5 \leq y < 0$, then $\frac{1}{y} \leq -\frac{1}{5}$.  If $0 < y \leq 4$, then $\frac{1}{y} \geq \frac{1}{4}$. Hence the range of $R$ is $\left(-\infty, -\frac{1}{5} \right] \cup \left[ \frac{1}{4}, \infty \right)$. 

\setcounter{HW}{\value{enumi}}
\end{enumerate}


\begin{multicols}{3}
\begin{enumerate}
\setcounter{enumi}{\value{HW}}

\item  $[-4,4]$ 
\item  $[-5,5)$
\item  none

\setcounter{HW}{\value{enumi}}
\end{enumerate}
\end{multicols}

\begin{multicols}{3}
\begin{enumerate}
\setcounter{enumi}{\value{HW}}

\item  $g(-2) = -5$
\item  none
\item  $(-2,-5)$, $(2,3)$

\setcounter{HW}{\value{enumi}}
\end{enumerate}
\end{multicols}

\begin{multicols}{3}
\begin{enumerate}
\setcounter{enumi}{\value{HW}}

\item  $[-2,2)$
\item  $[-4, -2]$, $(2,4]$
\item  $g(2) = 3$

\setcounter{HW}{\value{enumi}}
\end{enumerate}
\end{multicols}


\begin{multicols}{3}
\begin{enumerate}
\setcounter{enumi}{\value{HW}}

\item  $t=-2$
\item $(-4,0)$, $(0,0)$, $(4,0)$
\item  $(0,0)$

\setcounter{HW}{\value{enumi}}
\end{enumerate}
\end{multicols}

\begin{multicols}{3}
\begin{enumerate}
\setcounter{enumi}{\value{HW}}

\item  $-4$, $0$, $4$
\item  $[-4,0] \cup \{4\}$
\item $[-4,-2) \cup (-2.4]$

\setcounter{HW}{\value{enumi}}
\end{enumerate}
\end{multicols}

\begin{multicols}{3}
\begin{enumerate}
\setcounter{enumi}{\value{HW}}

\item $\{-4\} \cup (-2,0] \cup \{4\}$
\item  $5$
\item Neither.

\setcounter{HW}{\value{enumi}}
\end{enumerate}
\end{multicols}

\begin{enumerate}
\setcounter{enumi}{\value{HW}}
\addtocounter{enumi}{4}
\item \begin{enumerate}
\item Local maximum: $(0,1)$, no local minimum.  Increasing: $(0,2]$, decreasing: $[-2,0)$.
\item No local maximum,  local minimum: $(0,1)$.  Increasing: $[-2,0)$, decreasing: $(0,2]$.
\item No local maximum,  local minimum: $(0,-1)$.  Increasing: $[0,2]$, decreasing: $[-2,0]$.
\item Local maximum: $(0,5)$, no local minimum.  Increasing: $[-2,0]$, decreasing: $[0,2]$.
\end{enumerate}

\setcounter{HW}{\value{enumi}}
\end{enumerate}


